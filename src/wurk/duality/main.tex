\documentclass[a4paper, leqno]{article}

\makeatletter
\def\@arabic#1{\the\numexpr(#1)-1\relax}
\def\@roman#1{\romannumeral\numexpr(#1)-1\relax}
\def\@Roman#1{\expandafter\@slowromancap\romannumeral\numexpr(#1)-1\relax @}
\makeatother

\usepackage
  [ backend=biber
  , style=ieee
  , natbib=true
  ]
  {biblatex}

\DeclareFieldFormat{labelnumber}{\the\numexpr#1-1\relax}

\usepackage[margin=10mm]{geometry}
\usepackage{amsmath}
\usepackage[amsmath, standard, framed, hyperref, thmmarks, thref]{ntheorem}
\usepackage{mathtools}
  \DeclarePairedDelimiter{\parens}{\lparen}{\rparen}
  \DeclarePairedDelimiter{\bracks}{\lbrack}{\rbrack}
\usepackage{unicode-math}
  \unimathsetup{math-style=ISO, bold-style=ISO}
\usepackage{lualatex-math}

\setmainfont[Ligatures=TeX]{XITS}
\setmathfont{Asana Math}

\usepackage{titling}
  \setlength{\droptitle}{-36pt}
\usepackage{datetime2}
\usepackage[hidelinks]{hyperref}
\usepackage{cmll}
\usepackage{graphicx}

\usepackage{ebproof}
  \ebproofset{center=false}

\usepackage{mdframed}
\mdfsetup
  { linecolor=black
  , linewidth=1px
  , userdefinedwidth=.5\linewidth
  , startinnercode=\centering
  }

\newcommand*\qOrd{\symsfup{o}}
\newcommand*\qLin{\symsfup{l}}
\newcommand*\qAff{\symsfup{a}}
\newcommand*\qRel{\symsfup{r}}
\newcommand*\qInt{\symsfup{i}}

\newcommand*\pcon{\times}
\newcommand*\pconu{1}
\newcommand*\ncon{\mathbin{\&}}
\newcommand*\nconu{\top}
\newcommand*\pdis{+}
\newcommand*\pdisu{\bot}
\newcommand*\ndis{\mathbin{⅋}}
\newcommand*\ndisu{0}

\newcommand*\plimp{\mathbin{↙}}
\newcommand*\primp{\mathbin{↘}}
\newcommand*\nrimp{\mathbin{↖}}
\newcommand*\nlimp{\mathbin{↗}}

\newcommand*\pint{\mathord{!}}
\newcommand*\nint{\mathord{?}}
\newcommand*\plin{\mathord{¡}}
\newcommand*\nlin{\mathord{¿}}

\DeclarePairedDelimiter{\lrbrace}{\lbrace}{\rbrace}

\title{(co-)intuitionistic logics}
\author{sumi}
\date{\DTMnow}
\begin{document}
  %\maketitle

  let
  \begin{itemize}
    \item $φ,χ$ range over formulae
    \item $Φ,Χ,Ψ$ range over lists of formulae
    \item $\#Χ < 2$
    \item $m∈\lrbrace{\plin,\pint,\nlin,\nint}$
  \end{itemize}

  \noindent
  \begin{minipage}{.49\linewidth}
    \begin{mdframed}[frametitle=intuitionistic ordered logic]
      \begin{prooftree}
  \infer0[Id]{α⇒α}
\end{prooftree}
\\[\bigskipamount]
\begin{prooftree}
  \hypo{Φ,mφ,χ,Φ'⇒Χ}
  \infer1[EL$_0$]{Φ,χ,mφ,Φ'⇒Χ}
\end{prooftree}
\quad
\begin{prooftree}
  \hypo{Φ,φ,mχ,Φ'⇒Χ}
  \infer1[EL$_1$]{Φ,mχ,φ,Φ'⇒Χ}
\end{prooftree}
\hfill\null
\\[\bigskipamount]
\begin{prooftree}
  \hypo{Φ⇒Χ}
  \infer1[WL]{Φ,\pint φ⇒Χ}
\end{prooftree}
\hfill\null
\\[\bigskipamount]
\begin{prooftree}
  \hypo{Φ,\pint φ,\pint φ⇒Χ}
  \infer1[CL]{Φ,\pint φ⇒Χ}
\end{prooftree}
\hfill\null
\\[\bigskipamount]
\begin{prooftree}
  \hypo{Φ,φ,Φ'⇒Χ}
  \infer1[$\pint$L]{Φ,\pint φ,Φ'⇒Χ}
\end{prooftree}
\hfill
\begin{prooftree}
  \hypo{\pint Φ⇒φ}
  \infer[left label=$\pint$R]1{\pint Φ⇒\pint φ}
\end{prooftree}
\\[\bigskipamount]
\begin{prooftree}
  \hypo{Φ,φ,Φ'⇒Χ}
  \infer1[$\plin$L]{Φ,\plin φ,Φ'⇒Χ}
\end{prooftree}
\hfill
\begin{prooftree}
  \hypo{MΦ⇒φ}
  \infer[left label=$\plin$R]1{MΦ⇒\plin φ}
\end{prooftree}
\\[\bigskipamount]
\begin{prooftree}
  \hypo{Φ⇒φ}
  \hypo{Ψ,χ,Ψ'⇒Χ}
  \infer2[$\plimp$L]{Ψ,χ\plimp φ,Φ,Ψ'⇒Χ}
\end{prooftree}
\hfill
\begin{prooftree}
  \hypo{Φ,φ⇒χ}
  \infer[left label=$\plimp$R]1{Φ⇒χ\plimp φ}
\end{prooftree}
\\[\bigskipamount]
\begin{prooftree}
  \hypo{Φ⇒φ}
  \hypo{Ψ,χ,Ψ'⇒Χ}
  \infer2[$\primp$L]{Ψ,Φ,φ\primp χ,Ψ'⇒Χ}
\end{prooftree}
\hfill
\begin{prooftree}
  \hypo{φ,Φ⇒χ}
  \infer[left label=$\primp$R]1{Φ⇒φ\primp χ}
\end{prooftree}
\\[\bigskipamount]
\begin{prooftree}
  \hypo{Φ,Φ'⇒Χ}
  \infer1[$\pconu$L]{Φ,\pconu,Φ'⇒Χ}
\end{prooftree}
\hfill
\begin{prooftree}
  \infer[left label=$\pconu$R]0{⇒\pconu}
\end{prooftree}
\\[\bigskipamount]
\begin{prooftree}
  \hypo{Φ,φ,χ,Φ'⇒Χ}
  \infer1[$\pcon$L]{Φ,φ\pcon χ,Φ'⇒Χ}
\end{prooftree}
\hfill
\begin{prooftree}
  \hypo{Φ⇒φ}
  \hypo{Φ'⇒χ}
  \infer[left label=$\pcon$R]2{Φ,Φ'⇒φ\pcon χ}
\end{prooftree}
\\[\bigskipamount]
\hfill
\begin{prooftree}
  \infer[left label=$\nconu$R]0{Φ⇒\nconu}
\end{prooftree}
\\[\bigskipamount]
\begin{prooftree}
  \hypo{Φ,φ_i,Φ'⇒Χ}
  \infer1[$\ncon$L$_i$]{Φ,φ_0\ncon φ_1,Φ'⇒Χ}
\end{prooftree}
\hfill
\begin{prooftree}
  \hypo{Φ⇒φ}
  \hypo{Φ⇒χ}
  \infer[left label=$\ncon$R]2{Φ⇒φ\ncon χ}
\end{prooftree}
\\[\bigskipamount]
\begin{prooftree}
  \infer0[$\pdisu$L]{Φ,\pdisu,Φ'⇒Χ}
\end{prooftree}
\hfill\null
\\[\bigskipamount]
\begin{prooftree}
  \hypo{Φ,φ,Φ'⇒Χ}
  \hypo{Φ,χ,Φ'⇒Χ}
  \infer2[$\pdis$L]{Φ,φ\pdis χ,Φ'⇒Χ}
\end{prooftree}
\hfill
\begin{prooftree}
  \hypo{Φ⇒φ_i}
  \infer[left label=$\pdis$R$_i$]1{Φ⇒φ_0\pdis φ_1}
\end{prooftree}

    \end{mdframed}
  \end{minipage}
  \hfill
  \begin{minipage}{.49\linewidth}
    \begin{mdframed}[frametitle=co-intuitionistic ordered logic]
      \begin{prooftree}
  \infer[left label=Id]0{α⇒α}
\end{prooftree}
\iffalse
\\[\bigskipamount]
\begin{prooftree}
  \hypo{Χ⇒φ,Φ'}
  \hypo{φ⇒Φ}
  \infer[left label=Trans]2{Χ⇒Φ',Φ}
\end{prooftree}
\fi
\\[\bigskipamount]
\hfill
\begin{prooftree}
  \hypo{Χ⇒Φ',χ,mφ,Φ}
  \infer[left label=ER$_0$]1{Χ⇒Φ',mφ,χ,Φ}
\end{prooftree}
\quad
\begin{prooftree}
  \hypo{Χ⇒Φ',mχ,φ,Φ}
  \infer[left label=ER$_1$]1{Χ⇒Φ',φ,mχ,Φ}
\end{prooftree}
\\[\bigskipamount]
\hfill
\begin{prooftree}
  \hypo{Χ⇒Φ}
  \infer[left label=WR]1{Χ⇒\nint φ,Φ}
\end{prooftree}
\\[\bigskipamount]
\hfill
\begin{prooftree}
  \hypo{Χ⇒\nint φ,\nint φ,Φ}
  \infer[left label=CR]1{Χ⇒\nint φ,Φ}
\end{prooftree}
\\[\bigskipamount]
\begin{prooftree}
  \hypo{φ⇒\nint Φ}
  \infer1[$\nint$L]{\nint φ⇒\nint Φ}
\end{prooftree}
\hfill
\begin{prooftree}
  \hypo{Χ⇒Φ',φ,Φ}
  \infer[left label=$\nint$R]1{Χ⇒Φ',\nint φ,Φ}
\end{prooftree}
\\[\bigskipamount]
\begin{prooftree}
  \hypo{φ⇒MΦ}
  \infer1[$\nlin$L]{\nlin φ⇒MΦ}
\end{prooftree}
\hfill
\begin{prooftree}
  \hypo{Χ⇒Φ',φ,Φ}
  \infer[left label=$\nlin$R]1{Χ⇒Φ',\nlin φ,Φ}
\end{prooftree}
\\[\bigskipamount]
\begin{prooftree}
  \hypo{χ⇒φ,Φ}
  \infer1[$\nrimp$L]{φ\nrimp χ⇒Φ}
\end{prooftree}
\hfill
\begin{prooftree}
  \hypo{φ⇒Φ}
  \hypo{Χ⇒Ψ',χ,Ψ}
  \infer[left label=$\nrimp$R]2{Χ⇒Ψ',Φ,φ\nrimp χ,Ψ}
\end{prooftree}
\\[\bigskipamount]
\begin{prooftree}
  \hypo{φ⇒Φ}
  \hypo{Χ⇒Ψ',χ,Ψ}
  \infer2[$\nlimp$L]{Χ⇒Ψ',χ\nlimp φ,Φ,Ψ}
\end{prooftree}
\hfill
\begin{prooftree}
  \hypo{χ⇒Φ,φ}
  \infer[left label=$\nlimp$R]1{χ\nlimp φ⇒Φ}
\end{prooftree}
\\[\bigskipamount]
\begin{prooftree}
  \infer0[$\ndisu$L]{\ndisu⇒}
\end{prooftree}
\hfill
\begin{prooftree}
  \hypo{Χ⇒Φ',Φ}
  \infer[left label=$\ndisu$R]1{Χ⇒Φ',\ndisu,Φ}
\end{prooftree}
\\[\bigskipamount]
\begin{prooftree}
  \hypo{χ⇒Φ}
  \hypo{φ⇒Φ}
  \infer2[$\ndis$L]{χ\ndis φ⇒Φ}
\end{prooftree}
\hfill
\begin{prooftree}
  \hypo{Χ⇒Φ',χ,φ,Φ}
  \infer[left label=$\ndis$R]1{Χ⇒Φ',χ\ndis φ,Φ}
\end{prooftree}
\\[\bigskipamount]
\begin{prooftree}
  \infer0[$\pdisu$L]{\pdisu⇒Φ}
\end{prooftree}
\hfill\mbox{}
\\[\bigskipamount]
\begin{prooftree}
  \hypo{Χ⇒Φ',φ_i,Φ}
  \infer1[$\pdis$L$_i$]{Χ⇒Φ',φ_1\pdis φ_0,Φ}
\end{prooftree}
\hfill
\begin{prooftree}
  \hypo{χ⇒Φ}
  \hypo{φ⇒Φ}
  \infer[left label=$\pdis$R]2{χ\pdis φ⇒Φ}
\end{prooftree}
\\[\bigskipamount]
\hfill
\begin{prooftree}
  \infer[left label=$\nconu$R]0{φ⇒Φ',\nconu,Φ}
\end{prooftree}
\\[\bigskipamount]
\begin{prooftree}
  \hypo{φ_i⇒Φ}
  \infer1[$\ncon$L$_i$]{φ_1\ncon φ_0⇒Φ}
\end{prooftree}
\hfill
\begin{prooftree}
  \hypo{Χ⇒Φ',χ,Φ}
  \hypo{Χ⇒Φ',φ,Φ}
  \infer[left label=$\ncon$R]2{Χ⇒Φ',χ\ncon φ,Φ}
\end{prooftree}

    \end{mdframed}
  \end{minipage}

  \iffalse
  \noindent
  \begin{minipage}{.45\linewidth}
    \begin{definition}[intuitionistic linear logic]\leavevmode
      \begin{center}
        \begin{prooftree}
  \infer0[Refl]{A⇒A}
\end{prooftree}

\bigskip
\begin{prooftree}
  \hypo{Γ⇒A}
  \hypo{Γ',A⇒B}
  \infer2[Trans]{Γ,Γ'⇒B}
\end{prooftree}

\bigskip
\begin{prooftree}
  \hypo{Γ,A,B,Γ'⇒Δ}
  \infer1[EL]{Γ,B,A,Γ'⇒Δ}
\end{prooftree}
\hfill\mbox{}

\bigskip
\begin{prooftree}
  \hypo{Γ⇒Δ}
  \infer1[WL]{Γ,!A⇒Δ}
\end{prooftree}
\hfill\mbox{}

\bigskip
\begin{prooftree}
  \hypo{Γ,!A,!A⇒Δ}
  \infer1[CL]{Γ,!A⇒Δ}
\end{prooftree}
\hfill\mbox{}

\bigskip
\begin{prooftree}
  \hypo{Γ,A⇒Δ}
  \infer1[$!L$]{Γ,!A⇒Δ}
\end{prooftree}
\hfill
\begin{prooftree}
  \hypo{!Γ⇒A}
  \infer1[$!$R]{!Γ⇒!A}
\end{prooftree}

\bigskip
\begin{prooftree}
  \hypo{Γ⇒A}
  \hypo{Γ',B⇒Δ}
  \infer2[$→$L]{Γ,Γ',A→B⇒Δ}
\end{prooftree}
\hfill
\begin{prooftree}
  \hypo{Γ,A⇒B}
  \infer1[$→$R]{Γ⇒A→B}
\end{prooftree}

\bigskip
\begin{prooftree}
  \hypo{Γ⇒Δ}
  \infer1[$𝟏$L]{Γ,𝟏⇒Δ}
\end{prooftree}
\hfill
\begin{prooftree}
  \infer0[$𝟏$R]{⇒𝟏}
\end{prooftree}

\bigskip
\begin{prooftree}
  \hypo{Γ,A,B⇒Δ}
  \infer1[$⊗$L]{Γ,A⊗B⇒Δ}
\end{prooftree}
\hfill
\begin{prooftree}
  \hypo{Γ⇒A}
  \hypo{Γ'⇒B}
  \infer2[$⊗$R]{Γ,Γ'⇒A⊗B}
\end{prooftree}

\bigskip
\hfill
\begin{prooftree}
  \infer0[$⊤$R]{Γ⇒⊤}
\end{prooftree}

\bigskip
\begin{prooftree}
  \hypo{Γ,A_i⇒Δ}
  \infer1[$\&$L$_i$]{Γ,A_0\ampBin A_1⇒Δ}
\end{prooftree}
\hfill
\begin{prooftree}
  \hypo{Γ⇒A}
  \hypo{Γ⇒B}
  \infer2[$\&$R]{Γ⇒A\ampBin B}
\end{prooftree}

\bigskip
\begin{prooftree}
  \infer0[$𝟎$L]{Γ,𝟎⇒Δ}
\end{prooftree}
\hfill\mbox{}

\bigskip
\begin{prooftree}
  \hypo{Γ,A⇒Δ}
  \hypo{Γ,B⇒Δ}
  \infer2[$⊕$L]{Γ,A⊕B⇒Δ}
\end{prooftree}
\hfill
\begin{prooftree}
  \hypo{Γ⇒A_i}
  \infer1[$⊕$R$_i$]{Γ⇒A_0⊕A_1}
\end{prooftree}

\mbox{}

      \end{center}
    \end{definition}
  \end{minipage}
  \hfill
  \begin{minipage}{.45\linewidth}
    \begin{definition}[co-intuitionistic linear logic]\leavevmode
      \begin{center}
        \begin{prooftree}
  \infer0[Refl]{A⇒A}
\end{prooftree}

\bigskip
\begin{prooftree}
  \hypo{B⇒A,Γ'}
  \hypo{A⇒Γ}
  \infer2[Trans]{B⇒Γ',Γ}
\end{prooftree}

\bigskip
\hfill
\begin{prooftree}
  \hypo{Δ⇒Γ',B,A,Γ}
  \infer1[ER]{Δ⇒Γ',A,B,Γ}
\end{prooftree}

\bigskip
\hfill
\begin{prooftree}
  \hypo{Δ⇒Γ}
  \infer1[WR]{Δ⇒?A,Γ}
\end{prooftree}

\bigskip
\hfill
\begin{prooftree}
  \hypo{Δ⇒?A,?A,Γ}
  \infer1[CR]{Δ⇒?A,Γ}
\end{prooftree}

\bigskip
\begin{prooftree}
  \hypo{A⇒?Γ}
  \infer1[$?$L]{?A⇒?Γ}
\end{prooftree}
\hfill
\begin{prooftree}
  \hypo{Δ⇒A,Γ}
  \infer1[$?$R]{Δ⇒?A,Γ}
\end{prooftree}

\bigskip
\begin{prooftree}
  \hypo{A⇒B,Γ}
  \infer1[$←$L]{A←B⇒Γ}
\end{prooftree}
\hfill
\begin{prooftree}
  \hypo{Δ⇒B,Γ'}
  \hypo{A⇒Γ}
  \infer2[$←$R]{Δ⇒B←A,Γ',Γ}
\end{prooftree}

\bigskip
\begin{prooftree}
  \infer0[$⊥$L]{⊥⇒}
\end{prooftree}
\hfill
\begin{prooftree}
  \hypo{Δ⇒Γ}
  \infer1[$⊥$R]{Δ⇒⊥,Γ}
\end{prooftree}

\bigskip
\begin{prooftree}
  \hypo{B⇒Γ}
  \hypo{A⇒Γ}
  \infer2[$⅋$L]{B⅋A⇒Γ}
\end{prooftree}
\hfill
\begin{prooftree}
  \hypo{Δ⇒B,A,Γ}
  \infer1[$⅋$R]{Δ⇒B⅋A,Γ}
\end{prooftree}

\bigskip
\begin{prooftree}
  \infer0[$𝟎$L]{𝟎⇒Γ}
\end{prooftree}
\hfill\mbox{}

\bigskip
\begin{prooftree}
  \hypo{Δ⇒A_i,Γ}
  \infer1[$⊕$L$_i$]{Δ⇒A_1⊕A_0,Γ}
\end{prooftree}
\hfill
\begin{prooftree}
  \hypo{B⇒Γ}
  \hypo{A⇒Γ}
  \infer2[$⊕$R]{B⊕A⇒Γ}
\end{prooftree}

\bigskip
\hfill
\begin{prooftree}
  \infer0[$⊤$R]{A⇒⊤,Γ}
\end{prooftree}

\bigskip
\begin{prooftree}
  \hypo{A_i⇒Γ}
  \infer1[$\&$L$_i$]{A_1\ampBin A_0⇒Γ}
\end{prooftree}
\hfill
\begin{prooftree}
  \hypo{Δ⇒B,Γ}
  \hypo{Δ⇒A,Γ}
  \infer2[$\&$R]{Δ⇒B\ampBin A,Γ}
\end{prooftree}

\mbox{}

      \end{center}
    \end{definition}
  \end{minipage}

  \noindent
  \begin{minipage}{.45\linewidth}
    \begin{definition}[intuitionistic logic]\leavevmode
      \begin{center}
        \begin{prooftree}
  \infer0[Refl]{A⇒A}
\end{prooftree}

\bigskip
\begin{prooftree}
  \hypo{Γ⇒A}
  \hypo{Γ',A⇒B}
  \infer2[Trans]{Γ,Γ'⇒B}
\end{prooftree}

\bigskip
\begin{prooftree}
  \hypo{Γ,A,B,Γ'⇒Δ}
  \infer1[EL]{Γ,B,A,Γ'⇒Δ}
\end{prooftree}
\hfill\mbox{}

\bigskip
\begin{prooftree}
  \hypo{Γ⇒Δ}
  \infer1[WL]{Γ,A⇒Δ}
\end{prooftree}
\hfill\mbox{}

\bigskip
\begin{prooftree}
  \hypo{Γ,A,A⇒Δ}
  \infer1[CL]{Γ,A⇒Δ}
\end{prooftree}
\hfill\mbox{}

\bigskip
\begin{prooftree}
  \hypo{Γ⇒A}
  \hypo{Γ',B⇒Δ}
  \infer2[$→$L]{Γ,Γ',A→B⇒Δ}
\end{prooftree}
\hfill
\begin{prooftree}
  \hypo{Γ,A⇒B}
  \infer1[$→$R]{Γ⇒A→B}
\end{prooftree}

\bigskip
\begin{prooftree}
  \hypo{Γ⇒Δ}
  \infer1[$⊤$L]{Γ,⊤⇒Δ}
\end{prooftree}
\hfill
\begin{prooftree}
  \infer0[$⊤$R]{⇒⊤}
\end{prooftree}

\bigskip
\begin{prooftree}
  \hypo{Γ,A,B⇒Δ}
  \infer1[$∧$L]{Γ,A∧B⇒Δ}
\end{prooftree}
\hfill
\begin{prooftree}
  \hypo{Γ⇒A}
  \hypo{Γ'⇒B}
  \infer2[$∧$R]{Γ,Γ'⇒A∧B}
\end{prooftree}

\bigskip
\begin{prooftree}
  \infer0[$⊥$L]{Γ,⊥⇒Δ}
\end{prooftree}
\hfill\mbox{}

\bigskip
\begin{prooftree}
  \hypo{Γ,A⇒Δ}
  \hypo{Γ,B⇒Δ}
  \infer2[$∨$L]{Γ,A∨B⇒Δ}
\end{prooftree}
\hfill
\begin{prooftree}
  \hypo{Γ⇒A_i}∨
  \infer1[$∨$R$_i$]{Γ⇒A_0∨A_1}
\end{prooftree}

\mbox{}

      \end{center}
    \end{definition}
  \end{minipage}
  \hfill
  \begin{minipage}{.45\linewidth}
    \begin{definition}[co-intuitionistic logic]\leavevmode
      \begin{center}
        \begin{prooftree}
  \infer0[Refl]{A⇒A}
\end{prooftree}

\bigskip
\begin{prooftree}
  \hypo{B⇒A,Γ'}
  \hypo{A⇒Γ}
  \infer2[Trans]{B⇒Γ',Γ}
\end{prooftree}

\bigskip
\hfill
\begin{prooftree}
  \hypo{Δ⇒Γ',B,A,Γ}
  \infer1[ER]{Δ⇒Γ',A,B,Γ}
\end{prooftree}

\bigskip
\hfill
\begin{prooftree}
  \hypo{Δ⇒Γ}
  \infer1[WR]{Δ⇒A,Γ}
\end{prooftree}

\bigskip
\hfill
\begin{prooftree}
  \hypo{Δ⇒A,A,Γ}
  \infer1[CR]{Δ⇒A,Γ}
\end{prooftree}

\bigskip
\begin{prooftree}
  \hypo{A⇒B,Γ}
  \infer1[$←$L]{A←B⇒Γ}
\end{prooftree}
\hfill
\begin{prooftree}
  \hypo{Δ⇒B,Γ'}
  \hypo{A⇒Γ}
  \infer2[$←$R]{Δ⇒B←A,Γ',Γ}
\end{prooftree}

\bigskip
\begin{prooftree}
  \infer0[$⊥$L]{⊥⇒}
\end{prooftree}
\hfill
\begin{prooftree}
  \hypo{Δ⇒Γ}
  \infer1[$⊥$R]{Δ⇒⊥,Γ}
\end{prooftree}

\bigskip
\begin{prooftree}
  \hypo{B⇒Γ}
  \hypo{A⇒Γ}
  \infer2[$∨$L]{B∨A⇒Γ}
\end{prooftree}
\hfill
\begin{prooftree}
  \hypo{Δ⇒B,A,Γ}
  \infer1[$∨$R]{Δ⇒B∨A,Γ}
\end{prooftree}

\bigskip
\hfill
\begin{prooftree}
  \infer0[$⊤$R]{A⇒⊤,Γ}
\end{prooftree}

\bigskip
\begin{prooftree}
  \hypo{A_i⇒Γ}
  \infer1[$∧$L$_i$]{A_1∧A_0⇒Γ}
\end{prooftree}
\hfill
\begin{prooftree}
  \hypo{Δ⇒B,Γ}
  \hypo{Δ⇒A,Γ}
  \infer2[$∧$R]{Δ⇒B∧A,Γ}
\end{prooftree}

\mbox{}

      \end{center}
    \end{definition}
  \end{minipage}
  \fi

  \printbibliography[title=参考文献]
\end{document}